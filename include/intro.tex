
\chapter{Introduction}
\label{chap:intro}

Nuclear energy continues to be an important source of carbon-free
baseload power in countries around the world. Unlike many other
baseload power sources, nuclear reactors do not rely on
carbon-releasing fuels, but harness the energy released by the
natural splitting of very heavy atoms, such as Uranium and
Plutonium. This event, called fission, relies on the nucleus absorbing
a free neutral particle, called a neutron. The splitting then releases
more neutrons, resulting in further events. Early nuclear scientists
realized that, given enough material in the correct physical layout,
the reaction could be self-sustaining: a chain reaction.

This nuclear chain reaction is harnessed by modern nuclear reactors
to generate large amounts of energy. Neutrons are the key to starting
and maintaining the chain reaction, so understanding their energy and
spatial distribution is central to the study of reactors. Many methods
have been developed to understand this problem, one of which is the
Monte Carlo method. This method using stochastic sampling to simulate
the propagation and interactions of neutrons within the reactor.

The \gls{inl} has begun a program to restart a test reactor, the
\gls{treat} to assess the viability of advanced nuclear
fuels. Experiments performed by this reactor will
support the development and deployment of the next generation of
nuclear reactors by testing transient and accident conditions \cite{webtreat}.

\gls{inl} has developed advanced deterministic multi-group codes
within the MOOSE framework~\cite{gaston2009} to model the neutron
distribution in this reactor. These codes rely on Monte Carlo codes,
such as Serpent 2, 
% need a citation here.
to calculate the cross-sections used by the
multi-group solver~\cite{ortensi2016}. The fuel elements in the \gls{treat} reactor
contain a large amount of graphite, leading to a highly scattering
environment. Because it can be quite difficult to accurately represent
relatively large amounts of scattering, calculation of scattering cross-sections is of
% or some similar justification that doesn't assume the reader understands this connection.
vital importance. One of the methods proposed to improve this
cross-section generation is \gls{wdt}.

In this manuscript, we will discuss the
\hyperref[chap:background]{background} of Monte Carlo codes and
neutron propagation methods. We will then discuss the
\hyperref[sec:wdt]{new method, \gls{wdt}}, and describe an
\hyperref[sec:wdt_scattering]{extension of the method} to include
scattering. Finally, we will assess the
\hyperref[chap:results]{results} of three test-cases, to evaluate the
usefulness of the \gls{wdt} method and refine the parameters of its
use.
% link to the sections for each topic.



%%% Local Variables:
%%% mode: latex
%%% TeX-master: "../masters_report"
%%% End:
