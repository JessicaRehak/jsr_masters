
\chapter{Conclusions and Future Work}
\label{chap:conclusion}

Monte Carlo codes are an invaluable tool when modeling neutron
propagation. Our finite computing resources demand more efficient
algorithms for more complex geometries and problems. One such
algorithm with promising results is \gls{wdt}.  In this manuscript we
have presented an extension of this method to include scattering.  In
this section, we will discuss the conclusions that can be drawn from
the three test cases presented.

\section{Summary of results}
\label{sec:results_summary}
% I'd have a more introductory paragraph that brings us back to the larger context of what we're doing and why

For each test case, we examined three parameters: the infinite flux,
the total cross-section, and the scattering matrix for the zeroth
scattering moment. For the \gls{pwr} pin cell, we observed
improvements in the \gls{fom} for the thermal group in all three
parameters studied. A similar improvement was seen in the fast group
for the fast reactor pin cell. There is no indication of a clear trend
for the homogeneous fuel element.

Use of the \gls{wdt} method improved the \gls{fom} for the \gls{pwr}
and fast reactor pin cells in the groups that experience the most
absorption events. That is, the thermal group for the \gls{pwr} and
the fast group for the fast reactor. This is consistent with what we
expect from using the \gls{wdt} routine. With the original
delta-tracking routine, collisions that may result in absorption may
be discarded as virtual and therefore never scored. The \gls{wdt}
routine scores every collisions that results in absorption and reduces
the weight as necessary. This is a more efficient routine in energy
regions where absorption events are occurring at a high rate. The
additional scoring of absorption events improves the \gls{fom} for the
total cross-section; by not killing the neutrons at these events, the
\gls{fom} for flux is improved by survival biasing. 
% probably ok for here, but may want to emphasize some of these physical characeteristics in the problem description in the final paper.

In regions where a large amount of scattering occurs, the routine is
slightly less efficient. Instead of rejecting virtual collisions
outright, the rejection occurs after sampling the type of
collision. We observe the impact of this inefficiency in energy
regions where scattering dominates. The total cross-section and the
scattering matrix entries for the \gls{pwr} fast group both under
perform standard delta-tracking. The large amount of graphite in the
homogeneous fuel element makes it a highly scattering environment,
which may explain the under-performance in total cross-section of
\gls{wdt} in most cases.

We provide specific recommended values for the \gls{wdt} threshold in
the sections dedicated to the test cases. Overall, the method provides
some improvement in the \gls{fom} in circumstances where there is a
large amount of absorption, in line with what is expected of the method.

\section{Future Work}
\label{sec:future_work}

There is still a large amount of work to assess the usefulness and
performance of this method. Based on the results of this study, we
should assess test cases that have geometrically small regions of high
absorption, such as control rods. Also, test cases where
delta-tracking does not perform well, such as \gls{triso} particles,
should be analyzed. In addition to more test cases, adjusting the
upper boundary for ray tracing (see Fig.~\ref{fig:ray_wdt}) should be
analyzed. \Gls{wdt} may outperform ray tracing in some regions of very
low real collision probability. An expanded parametric study of
the \gls{wdt} threshold should be performed to investigate if the general
trend observed here holds. Such a study should include both a finer
examination of the \gls{wdt} threshold value, and examination of other
output parameters of interest.

\section{Conclusion}
\label{sec:conclusion}

In this manuscript, we have sought to determine the usefulness of the
\gls{wdt} method. We discussed the background of the Monte Carlo
simulation and how it is applied to neutral particle propagation. We
introduced Woodcock delta-tracking as a method to improve the
simulation in geometrically complex region. This algorithm was
modified by a recently introduced method, \acrlong{wdt}, which
replaced some virtual collisions with a weight reduction. We then
extended the method from considering only absorption events to include
scattering events. Finally, we examined three parameters of interest
from three test cases to assess the usefulness of \gls{wdt} and
determine a preferred threshold value for use of the algorithm. The
\gls{fom} improvements are in line with what was expected of the
\gls{wdt} method. The suggested future work should improve our
understanding of the impact of the method, and provide more robust
recommendations for future simulations.


%%% Local Variables:
%%% mode: latex
%%% TeX-master: "../masters_report"
%%% End:
